% !TeX program = pdfLaTeX
\documentclass[smallextended]{svjour3}       % onecolumn (second format)
%\documentclass[twocolumn]{svjour3}          % twocolumn
%
\smartqed  % flush right qed marks, e.g. at end of proof
%
\usepackage{amsmath}
\usepackage{graphicx}
\usepackage[utf8]{inputenc}

\usepackage[hyphens]{url} % not crucial - just used below for the URL
\usepackage{hyperref}
\providecommand{\tightlist}{%
  \setlength{\itemsep}{0pt}\setlength{\parskip}{0pt}}

%
% \usepackage{mathptmx}      % use Times fonts if available on your TeX system
%
% insert here the call for the packages your document requires
%\usepackage{latexsym}
% etc.
%
% please place your own definitions here and don't use \def but
% \newcommand{}{}
%
% Insert the name of "your journal" with
% \journalname{myjournal}
%

%% load any required packages here



% Pandoc citation processing


\begin{document}

\title{Perdiz arrow points from Caddo burial contexts aid in defining
discrete behavioral regions \thanks{Components of the analytical
workflow were developed and funded by a Preservation Technology and
Training grant (P14AP00138) to RZS from the National Center for
Preservation Technology and Training, as well as grants to RZS from the
Caddo Nation of Oklahoma, National Forests and Grasslands in Texas
(15-PA-11081300-033) and the United States Forest Service
(20-PA-11081300-074).} }



\author{  Robert Z. Selden 1 \and  John E. Dockall 2 \and  }

    \authorrunning{ Selden and Dockall }

\institute{
        Robert Z. Selden 1 \at
     Heritage Research Center and Department of Biology, Stephen F.
Austin State University; and Cultural Heritage Department, Jean Monnet
University \\
     \email{\href{mailto:zselden@sfasu.edu}{\nolinkurl{zselden@sfasu.edu}}}  %  \\
%             \emph{Present address:} of F. Author  %  if needed
    \and
        John E. Dockall 2 \at
     Cox\textbar McClain Environmental Consultants, Inc. \\
     \email{\href{mailto:johnd@coxmcclain.com}{\nolinkurl{johnd@coxmcclain.com}}}  %  \\
%             \emph{Present address:} of F. Author  %  if needed
    \and
    }

\date{Received: date / Accepted: date}
% The correct dates will be entered by the editor


\maketitle

\begin{abstract}
Recent research in the ancestral Caddo area has yielded evidence for
distinct \emph{behavioral regions}, across which material culture from
Caddo burials---Hickory Engraved and Smithport Plain bottles as well as
Gahagan bifaces---have been found to express significant morphological
differences. This inquiry assesses whether Perdiz arrow points from
Caddo burials, assumed to reflect design intent, may differ across the
same geography, and extend the pattern of shape differences to a third
category of Caddo material culture. Perdiz arrow points collected from
the geographies of the northern and southern Caddo communities of
practice defined in a recent social network analysis were employed to
test the hypothesis that morphological attributes differ, and are
predictable, between the two communities. Results indicate significant
between-community differences in maximum length, width, stem length, and
stem width, but not thickness. Using the same traditional metrics
combined with the tools of machine learning, a predictive
model---support vector machine---was designed to assess the degree to
which community differences could be predicted, achieving a receiver
operator curve score of 97 percent, and an accuracy score of 94 percent.
The subsequent geometric morphometric analysis identified significant
differences in Perdiz arrow point shape and size, coupled with
significant results for modularity and morphological integration. These
findings bolster the argument for the establishment of at least two
discrete \emph{behavioral regions} in the ancestral Caddo area defined
by discernible morphological differences across three categories of
Caddo material culture.
\\
\keywords{
        American Southeast \and
        Caddo \and
        Texas \and
        archaeoinformatics \and
        computational archaeology \and
        machine learning \and
        museum studies \and
        digital humanities \and
        STEM \and
    }


\end{abstract}


\def\spacingset#1{\renewcommand{\baselinestretch}%
{#1}\small\normalsize} \spacingset{1}


\hypertarget{intro}{%
\section{Introduction}\label{intro}}

Perdiz arrow points generally follow two distinct manufacturing
trajectories; one that enlists flakes, and the other, blade flakes
(Dockall, et al.~2020:I-120 - I-121; Johnson 1994:66-80; Ricklis
1994:213-214; Selden Jr, et al.~2021:2). Lithic tool stone in the Caddo
area of northeast Texas is relatively sparse, consists primarily of
chert, quartzite, and silicified wood characteristic of the local
geological formations, which may contribute to local variation in both
morphology and size (Banks 1990:Figure 2.1; Selden Jr, et al.~2021:
Figure 2). It has been demonstrated elsewhere that Perdiz arrow points
from northeast Texas vary significantly by time, raw material, and
burial context (Selden Jr, et al.~2021 and supplementary materials). In
outline, Perdiz arrow points possess a:

\begin{quote}
{[}t{]}riangular blade with edges usually quite straight but sometimes
slightly convex or concave. Shoulders sometimes at right angles to stem
but usually well barbed. Stem contracted, often quite sharp at base, but
may be somewhat rounded. Occasionally, specimen may be worked on one
face only or mainly on one face \ldots{} {[}w{]}orkmanship generally
good, sometimes exceedingly fine with minutely serrated blade edges
(Suhm, et al.~1954:504).
\end{quote}

A social network analysis of Historic Caddo (post-CE 1680) sites in
northeast Texas demonstrated two spatially distinct communities of
practice (Selden Jr.~2021a). The network analysis was limited to
Historic Caddo types; however, Formative Early Caddo (CE 800 -- 1200)
Gahagan bifaces and Caddo bottle types have been found to express
significantly different morphologies between the same two areas (Selden
Jr.~2018a, 2018b, 2019, 2021b), extending the temporal range of the
shape boundary. Gahagan bifaces from the ancestral Caddo area also
differ significantly in shape, size, and form from those recovered at
central Texas sites (Selden Jr., et al.~2020), suggestive of a second
shape boundary between the Caddo region and central Texas.

\hypertarget{sec:1}{%
\section{Section title}\label{sec:1}}

Text with citations by \cite{Galyardt14mmm}.

\hypertarget{sec:2}{%
\subsection{Subsection title}\label{sec:2}}

as required. Don't forget to give each section and subsection a unique
label (see Sect. \ref{sec:1}).

\hypertarget{paragraph-headings}{%
\paragraph{Paragraph headings}\label{paragraph-headings}}

Use paragraph headings as needed.

\begin{align}
a^2+b^2=c^2
\end{align}


\bibliographystyle{model5-names}
\bibliography{mybibfile.bib}

\end{document}
